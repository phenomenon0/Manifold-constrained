\section{Method}
\subsection{Error-Driven Sparse Routing}
For token \(x\), each expert \(e\) reconstructs \(\hat{x}_e\). Routing score is
the reconstruction error \(\mathrm{err}_e=\|x-\hat{x}_e\|^2\) (or configured
variant). MCQSMoE selects the top-k experts with lowest error:
\[
\mathcal{K}(x)=\operatorname{TopK}_{\min e}\ \mathrm{err}_e.
\]
Selected experts are combined by softmin weighting:
\[
w_e=\frac{\exp(-\mathrm{err}_e/T)}{\sum_{e'\in\mathcal{K}(x)}\exp(-\mathrm{err}_{e'}/T)},
\]
with fixed \(T=0.1\) in the current implementation.
Unlike standard learned-router MoE, this routing path is directly auditable:
the full error matrix, selected experts, weights, entropy, and residual norms
are exported by the audit object for each batch.

\subsection{Manifold-Constrained Shared-Anchor Expert Weights}
Each expert weight is represented as:
\[
W_e=\mathcal{R}(A+\Delta_e),
\]
where \(A\) is a layer-level shared anchor, \(\Delta_e\) is a per-expert
quantized delta (int4/int2 groupwise), and \(\mathcal{R}\) denotes manifold
retraction (row normalization on sphere rows in the tested configuration).
This avoids storing full dense weights per expert and enables explicit memory
control. In imported artifacts, the tested FFN path reports 1.94x memory
reduction versus naive per-expert fp16 storage.

\subsection{Execution Path}
Inference uses linear decomposition:
\[
Y = X(A+\Delta_e)^T = XA^T + X\Delta_e^T.
\]
The anchor term is computed once, then delta terms are computed for selected
experts and combined. Tokens are grouped by expert/mask assignment for batched
dispatch, reducing redundant work in sparse execution.

\subsection{Claim Compiler and Falsifiability}
The evaluation stack is organized as:
\[
\text{Claim}\rightarrow\text{Track}\rightarrow\text{Script}\rightarrow\text{Artifact}\rightarrow\text{Gate}.
\]
This structure enforces falsifiable claims: every statement in the paper must
map to a metric, an executable test path, and a concrete artifact.
\begin{figure}[t]
\centering
\fbox{%
\begin{minipage}{0.95\linewidth}
\small
\textbf{Protocol Flow:}\
Claim definitions (Protocol) $\rightarrow$ Track execution (Q/QS/P/PS/T)
$\rightarrow$ Scripted runs $\rightarrow$ Artifact logs
$\rightarrow$ Gate checks A--F $\rightarrow$ Summary output.

\vspace{0.4em}
\textbf{Strictness:}
missing trained artifacts or skip-detected Track T behavior triggers failure.
Missing benchmark coverage in PS triggers failure.
\end{minipage}%
}
\caption{Claim-to-artifact pipeline used in this paper package.}
\label{fig:protocol_flow}
\end{figure}

\begin{table*}[t]
\centering
\small
\begin{tabular}{p{0.24\linewidth}p{0.18\linewidth}p{0.26\linewidth}p{0.24\linewidth}}
\toprule
Claim & Primary Metric & Verification Route & Artifact Path \\
\midrule
Routing/composition determinism and stability
& Quality test pass/fail; routing audit stats
& Track Q + QS test suites
& \texttt{artifacts/logs/quality.log}, \texttt{artifacts/logs/quality\_multiseed.log} \\

Forward-path throughput gains for selected paths
& \texttt{ns/op}, \texttt{tok/s}, 95\% CI
& Track P + PS benchmarks
& \texttt{artifacts/logs/perf.log}, \texttt{artifacts/bench/perf\_multirun\_stats.csv} \\

Compression/quantization correctness within tolerance
& Golden parity errors; invariant checks; determinism
& Track T strict gate
& \texttt{artifacts/logs/track\_t.log} \\
\bottomrule
\end{tabular}
\caption{Claim-to-evidence mapping from protocol to reproducible artifacts.}
\label{tab:claim_to_evidence}
\end{table*}

