\section{Experimental Setup}
\subsection{Environment}
Environment metadata is captured in \texttt{artifacts/env/env\_snapshot.txt}:
Linux 6.17.12, Go 1.25.5, CPU AMD Ryzen 7 7700X (16 logical cores), and
31,184 MiB RAM.

\subsection{Evaluation Axes}
We evaluate three axes:
\begin{enumerate}
\item \textbf{Algorithmic validity:} correctness, determinism, routing behavior,
and trained-artifact parity (Q, QS, T).
\item \textbf{Systems efficiency:} throughput/latency on sparse forward paths
with repeated runs and confidence intervals (P, PS).
\item \textbf{Comparative sanity checks:} matched dense baseline and a fixed
external-style MCQA harness snapshot (DB, X).
\end{enumerate}

\subsection{Protocol Controls}
The canonical runner is:
\begin{verbatim}
./foundation_models/paper_mcsqoe/scripts/40_run_all.sh
\end{verbatim}
QS seeds are fixed (\texttt{11 23 47 101 211}). PS uses \(n=5\) repeated runs
and reports 95\% CI as \(1.96\cdot\sigma/\sqrt{n}\). DB aligns dense and QSMoE
parameter/compute budgets for direct micro-benchmark comparison.

\subsection{Scope Boundary}
This paper validates mechanism-level and systems-level claims for MCQSMoE.
It is not positioned as a full external leaderboard paper yet.
\begin{figure}[t]
\centering
\fbox{%
\begin{minipage}{0.95\linewidth}
\small
\textbf{Proven in this paper:} internal correctness tests, deterministic replay
for defined parity tests, benchmark throughput/latency statistics on one machine.

\vspace{0.4em}
\textbf{Not proven yet:} matched dense-vs-MCQSMoE external benchmark superiority,
public leaderboard quality claims, cross-hardware deployment guarantees.
\end{minipage}%
}
\caption{Claim boundary used to prevent over-interpretation of results.}
\label{fig:claim_boundary}
\end{figure}

