\section{Results}
\subsection{Overall Gate Status}
From \texttt{artifacts/reports/summary.md} (timestamp 2026-02-08T11:37:19Z):
Q=PASS, QS=PASS (5 seeds), P=PASS, PS=PASS (5 runs), DB=PASS, T=PASS, X=PASS.

\subsection{Sparse Execution Speedup Is Stable}
\begin{table*}[t]
\centering
\small
\begin{tabular}{lrrrrrr}
\toprule
Benchmark & n & Mean ns/op & Std ns/op & 95\% CI ns/op & Mean tok/s & 95\% CI tok/s \\
\midrule
ParallelForward\_medium\_b32/Parallel
& 5 & 65,762,102.8 & 7,455,173.7 & 6,534,747.9 & 491.02 & 42.84 \\
ParallelForward\_medium\_b32/ParallelZeroCopy
& 5 & 63,399,367.2 & 2,213,211.1 & 1,939,965.0 & 505.22 & 14.83 \\
ParallelForward\_medium\_b32/Serial
& 5 & 326,186,056.8 & 7,032,981.0 & 6,164,679.7 & 98.13 & 1.83 \\
QSMoEForward
& 5 & 9,716,165.6 & 380,163.8 & 333,228.3 & NA & NA \\
QSMoEForwardLarge
& 5 & 157,520,101.0 & 3,509,261.4 & 3,076,003.3 & NA & NA \\
WorkspaceForward\_medium\_b32
& 5 & 287,206,814.0 & 5,583,507.5 & 4,894,160.1 & NA & NA \\
\bottomrule
\end{tabular}
\caption{Canonical multi-run performance statistics from
\texttt{artifacts/bench/perf\_multirun\_stats.csv}.}
\label{tab:perf_multirun}
\end{table*}

Across five runs, \texttt{Parallel} achieves 529.7 tok/s versus 100.6 tok/s
for serial (5.26x), and \texttt{ParallelZeroCopy} achieves 512.8 tok/s
(5.10x). This indicates the sparse grouped-dispatch path preserves a large
speed advantage under repeated measurement, not only single-run peaks.

\subsection{QSMoE Beats Matched Dense Baseline in Tested Regimes}
\begin{table}[t]
\centering
\small
\begin{tabular}{lrrr}
\toprule
Case & QSMoE ns/op & Dense Matched ns/op & Dense/QSMoE \\
\midrule
small & 9,242,506 & 11,571,670 & 1.2520x \\
large & 152,577,554 & 182,059,712 & 1.1932x \\
\bottomrule
\end{tabular}
\caption{Dense baseline under matched parameter/compute budget
from \texttt{artifacts/reports/dense\_baseline\_summary.md}.}
\label{tab:dense_baseline}
\end{table}

Under matched parameter/compute settings, QSMoE is faster than dense on both
tested scales: 1.252x (small) and 1.193x (large), supporting the claim that
the sparse/compression-native path is practically useful, not only compact.

\subsection{Compression Fidelity and Determinism}
\begin{table}[t]
\centering
\small
\begin{tabular}{p{0.33\linewidth}p{0.5\linewidth}}
\toprule
Track T Check & Observed Result \\
\midrule
Required artifacts precheck & PASS (all required manifests and golden shard found) \\
Golden parity (layer 0) & max\_abs=0.000732, rmse=0.000185, nrmse=0.02\% \\
Golden parity (layer 5) & max\_abs=0.000784, rmse=0.000182, nrmse=0.02\% \\
Manifold invariant checks & max norm error range 0.000024 to 0.000069 \\
Determinism replay & PASS: identical outputs across 10 runs \\
Exported model memory summary & 12,672 KiB total; per-layer FFN savings 1.94x \\
Final gate status & \texttt{TRACK\_T\_STATUS=PASS} \\
\bottomrule
\end{tabular}
\caption{Track T artifact parity evidence from
\texttt{artifacts/logs/track\_t.log}.}
\label{tab:track_t}
\end{table}

Track T shows tight parity against exported golden artifacts, with
max absolute error below \(8\times10^{-4}\), low NRMSE, manifold invariant
checks passing, and exact deterministic replay across 10 runs.

\subsection{External Harness Snapshot (Calibration Checkpoint)}
\begin{table*}[t]
\centering
\small
\begin{tabular}{lrrrrrr}
\toprule
Model & N & Accuracy & ECE & Brier & AUROC & Confident-Wrong@0.8 \\
\midrule
\texttt{ar\_baseline} & 100 & 0.5100 & 0.0981 & 0.2294 & 0.3484 & 0.2222 \\
\texttt{diffusion\_baseline} & 100 & 0.3800 & 0.6192 & 0.6190 & 0.6378 & 0.6200 \\
\texttt{self\_consistency} & 100 & 0.7700 & 0.1888 & 0.1739 & 0.0816 & 0.0000 \\
\bottomrule
\end{tabular}
\caption{External-style MCQA harness snapshot (\texttt{mmlu-tiny}, seed 424242)
from \texttt{artifacts/reports/external\_eval/summary.md}.}
\label{tab:external_eval}
\end{table*}

The external snapshot is intentionally small-scale (\texttt{mmlu-tiny}, fixed
seed) and is used as an early calibration/behavior checkpoint, not a final
generalization claim.
